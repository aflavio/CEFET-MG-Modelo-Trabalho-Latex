% -----------------------------------------------------------------------------
%   Arquivo: ./02-elementos-textuais/metodologia.tex
% -----------------------------------------------------------------------------



\chapter{Metodologia}
\label{chap:metodologia}
Cada capítulo deve conter uma pequena introdução (tipicamente, um ou dois parágrafos), em seção não numerada, que deve deixar claro o objetivo e o que será discutido no capítulo, bem como a organização do capítulo.



\section{Delineamento da pesquisa}
\label{sec:titSecDelPesq}

Inserir seu texto aqui...



\section{Coleta e tratamento de dados}
\label{sec:titSecColDad}

Inserir seu texto aqui...

Exemplo de tabela:

\input{./05-tabelas/tabteste}



\section{Equações}
\label{sec:equacoes}

A transformada de Laplace é dada na \autoref{eq:laplace}, enquanto a Eq. \ref{eq:dft} apresenta a formulação da transformada discreta de Fourier bidimensional\footnote{Deve-se reparar na formatação esteticamente perfeita destas equações.}. Observe que utilizamos propositalmente duas formas distintas para referenciar as equações.

\begin{equation}
	X(s) = \int\limits_{t = -\infty}^{\infty} x(t) \, \text{e}^{-st} \, dt
	\label{eq:laplace}
\end{equation}

\begin{equation}
	F(u, v) = \sum_{m = 0}^{M - 1} \sum_{n = 0}^{N - 1} f(m, n) \exp \left[ -j 2 \pi \left( \frac{u m}{M} + \frac{v n}{N} \right) \right]
	\label{eq:dft}
\end{equation}



\section{Algoritmos}\label{sec:algoritmos}

Os algoritmos devem ser feitos segundo o modelo abaixo. Para isso, utilizar o pacote {\ttfamily algorithm2e} no início do arquivo principal como neste exemplo.
\\
\\

\input{./07-algoritmos/vertices}
