%
% Modelo de trabalho acadêmico para disciplinas em português brasileiro e em 
% conformidade com as normas da ABNT. Melhor visualizado em 80 colunas.
%
% Documento principal
% Data: 18 de março de 2016
%
% *****************************************************************************
% *  Centro Federal de Educação Tecnológica de Minas Gerais - CEFET-MG        *
% *                                                                           *
% *                                                                           *
% *  Autor que adaptou : Augusto Morais <aflavio@gmail.com>                   * 
% *  Source: https://github.com/aflavio/cefet-MG-latex                         
% *                                                                           *
% *                                                                           *
% *  Autor: Henrique E. Borges <henrique@lsi.cefetmg.br>                      *
% *  Autor: Denise de Souza <densouza@gmail.com>                              *
% *  Autor: Cristiano Fraga G. Nunes <cfgnunes@gmail.com>                     *
% *  Autor: Lauro César <https://code.google.com/p/abntex2/                   *
% *                                                                           *
% *****************************************************************************



\documentclass[
      %twoside,         % Impressão em frente (anverso) e verso. Oposto a oneside
      oneside,          % Impressão apenas no anverso. Oposto a twoside
      a4paper,          % Tamanho do papel
      english,          % Idioma adicional para hifenização
      brazilian,        % O ultimo idioma será o principal idioma do documento    
]{abntex2-cefetmg} 



% -----------------------------------------------------------------------------
%    Configura as citações bibliográficas conforme a norma ABNT
% -----------------------------------------------------------------------------

\usepackage[brazilian,hyperpageref]{backref}
\usepackage[alf, 
abnt-emphasize=bf, 
bibjustif, 
recuo=0cm,
abnt-doi=expand,        % Expande um endereço com doi: para http://dx.doi.org/
abnt-url-package=url,   % utiliza o pacote url
abnt-refinfo=yes,       % utiliza o estilo bibliográfico abnt-refinfo
abnt-etal-cite=3, 
abnt-etal-list=3,
abnt-thesis-year=final]{abntex2cite}    % Formata as citações conforme ABNT


% -----------------------------------------------------------------------------
%    Pacotes utilizados 
% -----------------------------------------------------------------------------

\usepackage[utf8]{inputenc}     % Usa codificação UTF-8
\usepackage[T1]{fontenc}        % Seleção de código de fonte

% Fonts: Times, LAtin Modern, Palatino, Latin Modern, Helvetica
\usepackage[scaled]{helvet}		  
\usepackage{scalefnt}			      
%\usepackage{times}				      
%\usepackage{lmodern}			      
%\usepackage{palatino}			    
%\usepackage{lmodern}			      

\renewcommand*\familydefault{\sfdefault} 	

% Inserção de caracteres gregos, matematicos, etc
\usepackage{upgreek}			      
\usepackage{latexsym}			      
\usepackage{amsfonts, amssymb, amsmath, dsfont}		


\usepackage{lscape}             % Páginas em modo "paisagem"
\usepackage{indentfirst}        % Indenta o primeiro parágrafo de cada seção.
\usepackage{microtype}          % Melhora a justificação do documento
\usepackage{hyperref}           % Usado para criar “hyperlinks” no PDF
\usepackage{breakurl}           % Permite quebra de linha em URLs longas
\usepackage{graphicx}           % Inclusão de gráficos e figuras
\usepackage{tikz}               % Pacote para desenhos
\usepackage{color, colortbl}    % Controle das cores
\usepackage[bottom]{footmisc}   % Notas de rodapé sempre na mesma posição
\usepackage{bookmark}           % Cria menu de bookmarks
\usepackage{verbatim}           % Apresenta texto tal como escrito no doc.
\usepackage{multirow, array}    % Permite tabelas com múltiplas linhas e colunas
\usepackage{booktabs}           % Réguas horizontais em tabelas
\usepackage{float}              % Tabelas/figuras em ambiente multi-colunas
\usepackage{subeqnarray}        % Permite subnumeracao de equações

\usepackage[pointedenum]{paralist}              % Listas numeradas/bullets
\usepackage[algoruled, portuguese]{algorithm2e} % algoritmos em Português

\usepackage{makeidx}            % Usado para produzir índice remissivo (glossário)
	\makeindex                  % Compila o índice
	
	
% -----------------------------------------------------------------------------	
%   Pacotes DESABILITADOS POR DEFAULT
% -----------------------------------------------------------------------------
%\usepackage[hyphens]{url}  % Melhora apresentação de URLs
%\usepackage{url}           % Melhora apresentação de URLs
%\usepackage{lettrine}      % Primeira letra do início de um texto é aumentada
%\usepackage{balance}       % Balanceia o texto no artigo
%\usepackage{nomencl}       % Usado para produzir lista de símbolos	
%\usepackage{subfig}        % Permite posicionar figuras
%\usepackage{picinpar}      % Permite posicionar imagens em parágrafos
%\usepackage{psfrag}        % Inclusão de símbolos latex em figuras eps



% Insere e constroi alguns elementos pré-textuais

\include{./01-elementos-pre-textuais/capa}

% -----------------------------------------------------------------------------
%   Configurações de aparência do PDF final
% -----------------------------------------------------------------------------

\definecolor{blue}{RGB}{13,71 ,161}	% Paleta de cor do Google Material Design

\makeatletter
\hypersetup{
	portuguese,
  colorlinks=true,  
	linkcolor=blue,			
	citecolor=blue,			
	filecolor=blue,			
	urlcolor=blue, 			
	breaklinks=true,
	pdftitle={\@title},
	pdfauthor={\@author},
	pdfkeywords={abnt, latex, abntex, abntex2}
}
\makeatother


% -----------------------------------------------------------------------------
%   Hifenização de palavras não constantes do dicionário
% -----------------------------------------------------------------------------

\hyphenation{
		qua-dros-cha-ve
		Bras-nett
		Kat-sa-gge-los
}


% -----------------------------------------------------------------------------
%   Inclui todos os arquivos do trabalho acadêmico
% -----------------------------------------------------------------------------

\begin{document}

\frenchspacing    % Retira o espaço extra desnecessário nas frases


% Elementos pré-textuais: CAPA
\pretextual
\imprimircapa
\include{./01-elementos-pre-textuais/sumario}		

% Insere os elementos textuais
\textual
% -----------------------------------------------------------------------------
%   Arquivo: ./02-elementos-textuais/introducao.tex
% -----------------------------------------------------------------------------



\chapter{Introdução}
\label{chap:introducao}

O presente trabalho tem por objetivo implementar os algoritmos MaxMin1 e MaxMin2 para verificar a complexidade para busca do maior e menor elementos de vários vetores não ordenados. 

Além disso, faz-se necessário discutir, ao final, discutir os dados obtidos.





			
% -----------------------------------------------------------------------------
%   Arquivo: ./02-elementos-textuais/metodologia.tex
% -----------------------------------------------------------------------------



\chapter{Metodologia}
\label{chap:metodologia}
O trabalho foi implementado na linguagem Python, versão 3.5 e interpretado no Linux Debian - Stretch 64 bits. 

Foram utilizados os seguintes parâmetros: 

\begin{table}[!htb]
	\centering
	\caption[Parametros utilizados]{Parametros utilizados.
	\label{tab:parametros}}
	\begin{tabular}{rrrrr}
		\toprule
			& Vetores & Tamanho de cada vetor & Valores aleatórios &  \\
		\midrule
			MaxMin1 & 500 & 10.000 & 0 a 1x10\textsuperscript{7} \\
			MaxMin2 & 500 & 10.000 & 0 a 1x10\textsuperscript{7} \\
		\bottomrule
	\end{tabular}
\end{table}



Para efeito ilustrativo, segue o código de implementação MaxMin1 em Python:\\

\begin{python}
def MaxMin1(A):
	Max = A[0]
	Min = A[0]
	j = int()
	
	for i in range(1, len(A)):
	
		j += 1 % Contador 
		if A[i] > Max:
		Max = A[i]
		
		j += 1 % Contador 
		if A[i] < Min:
		Min = A[i]
	
	return j 

\end{python}
		

\pagebreak

Já para o algoritmo Maxmin2, temos: \\

\begin{python}
def MaxMin2(A):
    Max = A[0]
    Min = A[0]
    j = int()

    for i in range(1, len(A)):

        j += 1 % Contador
        if A[i] > Max:
            Max = A[i]
            j += 1
        elif A[i] < Min:
            Min = A[i]

    return j % Retorna Contador
\end{python}



			
% -----------------------------------------------------------------------------
%   Arquivo: ./02-elementos-textuais/resultados.tex
% -----------------------------------------------------------------------------



\chapter{Análise e Discussão dos Resultados}

Como era de se esperar, o algoritmo MaxMin1 apresentou uma complexidade de 2(n-1), onde n é o tamanho do vetor, no caso, 10.000. Portanto, como ilustrado no gráfico abaixo, temos uma linha reta. Já para o algoritmo MaxMin2, percebemos uma performance melhor, uma vez que não temos a "comparação" de todos os elementos do vetor para encontrar o Maior e Menor valor. 

Segue a ilustração abaixo. 

\begin{figure}[!htb]
	\centering
	\caption{MaxMin1 x MaxMin2}
	\includegraphics[width=0.8\textwidth]{./04-figuras/MaxMin1xMaxMin2.pdf}
	\fonte{Própria}
	\label{fig:maxmin1vsmaxmin2}
\end{figure}

\pagebreak
Apesar de não visivel (devido a escala do gráfico), o método MaxMin2 apresentou uma pequena variação de seus contadores. Esta pequena variação pode ser visualizada no gráfico abaixo onde a escala foi mudada. 


\begin{figure}[!htb]
	\centering
	\caption{MaxMin2}
	\includegraphics[width=0.8\textwidth]{./04-figuras/MaxMin2.pdf}
	\fonte{Própria}
	\label{fig:maxmin2}
\end{figure}



O valor médio encontrado pelo algoritmo MaxMin1 e MaxMin2 estão ilustrados na tabela a seguir: 

\begin{table}[!htb]
	\centering
	\caption[Médias encontradas]{Médias encontradas.
	\label{tab:medias}}
	\begin{tabular}{rrrrr}
		\toprule
			& Média &  \\
		\midrule
			MaxMin1 & 19998.00 \\
			MaxMin2 & 10007.74 \\
		\bottomrule
	\end{tabular}
\end{table}


		
\include{./02-elementos-textuais/conclusao}				

%   Insere os elementos pós-textuais
\postextual
\include{./03-elementos-pos-textuais/referencias}


\end{document}
